% XeLaTeX can use any Mac OS X font. See the setromanfont command below.
% Input to XeLaTeX is full Unicode, so Unicode characters can be typed directly into the source.

% The next lines tell TeXShop to typeset with xelatex, and to open and save the source with Unicode encoding.

%!TEX TS-program = xelatex
%!TEX encoding = UTF-8 Unicode

\documentclass[11pt]{article}

\usepackage [brazil]{babel}           % nomes e hifenação em português
\usepackage{hyperref}
\usepackage[
  right=0.75in,
  left=0.75in,
  top=1in,
  bottom=1in]{geometry}            % See geometry.pdf to learn the layout options. There are lots.
\geometry{a4paper}                   % ... or a4paper or a5paper or ... 
%\geometry{landscape}                % Activate for for rotated page geometry
%\usepackage[parfill]{parskip}    % Activate to begin paragraphs with an empty line rather than an indent
\usepackage{graphicx}
\usepackage{amssymb}

% Will Robertson's fontspec.sty can be used to simplify font choices.
% To experiment, open /Applications/Font Book to examine the fonts provided on Mac OS X,
% and change "Hoefler Text" to any of these choices.

\usepackage{fontspec,xltxtra,xunicode}
\defaultfontfeatures{Mapping=tex-text}
\setromanfont[Mapping=tex-text]{Gill Sans}
\setsansfont[Scale=MatchLowercase,Mapping=tex-text]{Gill Sans}
\setmonofont[Scale=MatchLowercase]{Monaco}

% Don't indent paragraphs.
\setlength\parindent{0em}

% Set your name here
\def\name{\textbf{Marcelo} de Brito Soares \textbf{Almeida}}

% Set you CV Link here
\def\footerlink{https://github.com/marcelopazzo/CV}

% Replace this with a link to your CV if you like, or set it empty
% (as in \def\footerlink{}) to remove the link in the footer:
\def\footerlink{https://github.com/marcelopazzo/CV}

% The following metadata will show up in the PDF properties
\hypersetup{
  colorlinks = true,
  urlcolor = black,
  pdfauthor = {\name},
  %pdfkeywords = {computer science, software, development},
  pdftitle = {\name: Curriculum Vitae},
  pdfsubject = {Curriculum Vitae},
  pdfpagemode = UseNone
}

% Customize page headers
\pagestyle{myheadings}
\markright{\name}
\thispagestyle{empty}

% Don't indent paragraphs.
\setlength\parindent{0em}

%\let\oldsubsbsection\subsection
%\renewcommand{\subsection}{\addtolength{\leftskip}{5mm}\oldsubsection}

\begin{document}

\begin{flushright}

{\huge \name}

% Alternatively, print name centered and bold:
%\centerline{\huge \bf \name}

\vspace{0.05in}
(85) 8122-5610 / \href{mailto:marcelopazzo@gmail.com}{\tt marcelopazzo@gmail.com} / skype: \tt marcelopazzo

\end{flushright}

\section*{Perfil}
\begin{itemize}
  \item Bacharelado em Ciência da Computação na Universidade Federal do Ceará (concluído em 2008)
  \item Desenvolvimento em plataforma Java (J2SE)
  \item Desenvolvimento em plataforma .NET / Mono em C\#
  \item Outras linguagens: Java, Ruby, Python, PHP, C, C++, Objective-C, Scheme, Shell
  \item Bancos de dados: Oracle, MySQL, SQLite
  \item Experiência em ambientes Linux, Solaris e Mac OS X
  \item Experiência com desenvolvimento Android - Aplicação publicada no Android Market: \\*
            \href{http://marcelopazzo.com/glasgow}{\tt http://marcelopazzo.com/glasgow}
  \item Experiência com integração contínua usando TeamCity, Maven e Nexus
  \item Experiência em testes com os framworks JUnit e mockito
  \item Sistemas de controle de versão: SVN e Git
  \item Experiência com metodologia de desenvolvimento ágil (Scrum)
  \item Servidor de troca de mensagens IBM WebSphere MQ
  \item Idiomas estrangeiros: Inglês (fluente)
\end{itemize}

\section*{Experiência profissional}
{\addtolength{\leftskip}{3.5mm}\subsection*{CPQi (Janeiro/2009 - Presente)}
\begin{itemize}
\item Lider técnico de dois projetos para desenvolvimento de customizações para o sistema Calypso (Java e Oracle)
\item Desenvolvimento de aplicações em Ruby on Rails para duas grandes empresas de apostas da Europa 
\end{itemize}

\subsection*{SecrelNet Internet (Outubro/2007 - Janeiro/2009)}
\begin{itemize}
\item Desenvolvimento de \textit{middlewares} e \textit{web services} para automatização e administração remota de servidores em Shell e C\#, rodando com Mono.
\item Desenvolvimento de aplicações PHP usando o framework Symfony
\item Manutenção do Webmail dos clientes
\end{itemize}

\subsection*{Hospital Universitário Walter Cantídio - UFC (Junho/2006 - Novembro/2006)}
\begin{itemize}
\item Criação do site do Hospital (PHP + MySQL)
\end{itemize}

\bigskip

% Footer
\begin{center}
  \begin{footnotesize}
    Última atualização: \today \\
    \tt{\footerlink}
  \end{footnotesize}
\end{center}

\end{document}  